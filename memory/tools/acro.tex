\DeclareAcronym{GPOS}{
  short = GPOS,
  long  = {General Purpose Operating System},
  tag = abbrev,
}
\DeclareAcronym{VM}{
  short = VM,
  long  = {Virtual Memory, a subsystem of the Linux kernel that manages memory allocation},
  tag = abbrev,
}
\DeclareAcronym{cgroups}{
  short = cgroups,
  long  = {Control Groups, the primordial feature of the Linux kernel for restricting, distributing \& auditing access to system resources among processes},
  tag = abbrev,
}
\DeclareAcronym{memcg}{
  short = memcg,
  long  = {the controller (or kind of cgroup) that governs usage of memory},
  tag = abbrev,
}
\DeclareAcronym{VFS}{
  short = VFS,
  long  = {Virtual File System, a subsystem of the Linux kernel that manages mounted filesystems and operations upon files},
  tag = abbrev,
}
\DeclareAcronym{blkio}{
  short = blkio,
  long  = {the controller (or kind of cgroup) that governs bandwidth \& latency regarding block I/O},
  tag = abbrev,
}
\DeclareAcronym{BIO}{
  short = BIO,
  long  = {Block I/O, a request to operate on a block storage device; also refers to the subsystem that manages block storage},
  tag = abbrev,
}
\DeclareAcronym{UML}{
  short = UML,
  long  = {User Mode Linux, a mechanism (or special architecture) that allows a kernel to run as a regular user-space application within another kernel},
  tag = abbrev,
}
\DeclareAcronym{FCFS}{
  short = FCFS,
  long  = {First Come First Serve, a simple scheduling policy that executes requests in order of arrival},
  tag = abbrev,
}
\DeclareAcronym{inode}{
  short = inode,
  long  = {a structure holding metadata about a node in a filesystem (a regular file, directory, symlink, device, pipe or socket)},
  tag = abbrev,
}
\DeclareAcronym{dentry}{
  short = dentry,
  long  = {Directory Entry, a structure holding the name, kind and pointed inode for an entry in a directory},
  tag = abbrev,
}
\DeclareAcronym{task}{
  short = task,
  long  = {subject of CPU scheduling; this generally refers to user-space threads, but in the general sense includes kernel worker tasks as well},
  tag = abbrev,
}
\DeclareAcronym{BPF}{
  short  = BPF,
  long   = {Berkeley Packet Filter, a language \& virtual machine definition},
  tag = abbrev,
}
\DeclareAcronym{BFQ}{
  short  = BFQ,
  long   = {Budget Fair Queueing, a proportional-share I/O scheduler of the Linux kernel},
  tag = abbrev,
}
\DeclareAcronym{netlink}{
  short  = netlink,
  long   = {mechanism \& transport protocol for user-space processes to communicate with the Linux kernel, usually to perform administrative operations. Used for complex APIs as an alternative to ioctl or filesystem-based interfaces},
  tag = abbrev,
}
\DeclareAcronym{PoC}{
  short  = PoC,
  long   = {Proof of Concept},
  tag = abbrev,
}
\DeclareAcronym{FFI}{
  short  = FFI,
  long   = {Foreign Function Interface, a mechanism that allows a programming language to dynamically call routines from another},
  tag = abbrev,
}
